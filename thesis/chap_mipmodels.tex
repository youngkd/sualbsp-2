%!TEX root = thesis-kdyoung.tex

% \begin{savequote}%[45mm]
% ---The formulation of the problem is often more essential than its solution
% \qauthor{Albert Einstein}
% \end{savequote}

\chapter{Mixed-Integer Programming Formulation}
\label{chap:mip}
This chapter presents three possible mixed-integer programs to model
the \sua{2}.
We adapt the work of \authciteb{Esmaeilbeigi2016}, in which
three similar formulations were presented.
These programs will provide a benchmark against which
to test our solution methodology presented in the following
chapter.
\citeauthor{Esmaeilbeigi2016} gave possible valid inequalities
which could be added to these formulations, however these 
additional constraints are as yet untested by the literature.

One of the research aims we have for our project is to test and compare
our solution methodology against those previously presented
in the literature.
However, due to the limited research that has
been done into the type-2 variant of the \sua{}, there is little
work to compare our approach against.
The only computational results available which are related to the
\sua{2} are by \authciteb{Yolmeh2012}.
For reasons detailed in Section \ref{sec:exp:data}, we chose to test
on a different set of data to that of \citeauthor{Yolmeh2012},
which meant that there is currently no computational
results that can be used to make a direct comparison.
It is possible to roughly judge the quality our results against those of \citeauthor{Yolmeh2012}
by comparing how our solution procedure performs on instances
of similar size.
However, to gain an exact measurement of the quality of different
approaches, we were motivated to implement and test the mixed-integer programs presented
in \citeauthor{Esmaeilbeigi2016}.

In order to capture a richer structure of the \sua{},
Table \ref{tab:mipParams} expands upon the notation that was defined earlier in 
Section \ref{sec:intro:probDef}. The work of \citeauthor{Esmaeilbeigi2016}
provides the basis of the notation we formulated for our central thesis.
Our choice of big-$M$ value listed in Table \ref{tab:mipParams} 
is calculated as the maximum possible cycle time value, 
which is detailed later in Equation \ref{eq:mip:cUB}.

\begin{table}[tbp]
	\def\arraystretch{1.1}
	\centering
	\caption{Notation Summary}
	\vspace{2mm}
	\begin{tabular}{lp{0.8\textwidth}}
		\toprule
		Notation & Definition  \\\midrule\midrule
		$n$ & The number of tasks\\
		$m$ & The number of stations\\
		$V$ & Set of tasks, indexed by $i$, $j$ and $v$ \\
		$K$ & Set of stations, indexed by $k$ \\
		$E (E^*)$ & Set of direct (all) $(i,j)$ precedence relations\\
		$t_i$ & Processing time of task $i$ \\
		$t_{\text{sum}}$ & Sum of all processing times, $\sum_{i\in V}t_i$ \\
		$P_i (P_i^*)$ & Set of direct (all) predecessors of task $i \in V$ \\
		$F_i (F_i^*)$ & Set of direct (all) successors of task $i \in V$ \\
		$\phi_{ij}$ & Setup time of task $j \in V$ when performed immediately after task $i \in V$ in the forward station load\\
		$\beta_{ij}$ & Setup time of task $j \in V$ when performed immediately after task $i \in V$ in the backward station load\\
		$F_i^\phi$ & Set of tasks which may directly follow task $i \in V$ in the forward station load,
					$F_i^\phi=\{\:j\in V- (F_i^*-F_i)-P_i^*-\{i\}\:|\: FS_i\cap FS_j \neq \emptyset\:\}$\\
		$F_i^\beta$ & Set of tasks which may directly follow task $i \in V$ in the backward station load,
					$F_i^\beta=\{\:j\in V- F_i^*\:|\: FS_i\cap FS_j \neq \emptyset\:\}$.\\
		$P_i^\phi$ & Set of tasks which may directly precede task $i \in V$ in the forward station load,
					$P_i^\phi=\{\: j \in V \:|\: i \in F_j^\phi \:\}$ \\
		$P_i^\beta$ & Set of tasks which may directly precede task $i \in V$ in the backward station load,
					$P_i^\beta=\{\: j \in V \:|\: i \in F_j^\beta \:\}$ \\
		$FS_i$ & Set of stations which task $i \in V$ can be feasibly assigned\\
		$FT_k$ & Set of tasks which can be feasibly assigned to station $k \in K$\\
		$\bar{c}(\ul{c})$ & Upper (lower) bound on the cycle time \\
		$M$ & Sufficiently large ``big-$M$'' value, $M=t_{sum} + \beta_{n,1} + \sum_{i=1}^{n-1} \phi_{i,i+1} $ \\
		\bottomrule
	\end{tabular}
	\label{tab:mipParams}
\end{table}

To illustrate how this notation can define the input properties of the
\sua{2}, we return to the simple instance of the problem 
in Example \ref{ex:mip:notationIllust} originally presented
in Chapter \ref{chap:intro}.
\begin{example}\label{ex:mip:notationIllust}
	Again, consider the instance of the \sua{2} as defined
	previously in Example \ref{ex:intro:simpleSetup},
	with setup times defined in Tables \ref{tab:intro:forwSetupTimes}
	and \ref{tab:intro:backSetupTimes} (see Chap. \ref{chap:intro}).
	This simple example has input parameters defined as follows,
	\begin{IEEEeqnarray}{rClCrCl}
		V &=& \{1,2,3,4=n\}, &\hspace{4mm}& K &=& \{1,2,3=m\}, \nonumber\\[\eqnv]
		E &=& \big\{(1,2),(2,3),(2,4)\big\}, & & t&=&(6,5,2,9), \nonumber\\[\eqnv]
		F_i^\phi &=& \big(\{2\},\{3,4\},\{4\},\{3\}\big), & & F_i^\beta&=&\big(\{1\},\{1,2\},\{1,2,3,4\},\{1,2,3,4\}\big), \nonumber\\[\eqnv]
		P_i^\phi &=& \big(\emptyset,\{1\},\{2,4\},\{2,3\}\big), & & P_i^\beta&=&\big(\{1,2,3,4\},\{2,3,4\},\{3,4\},\{3,4\}\big), \nonumber\\[\eqnv]
		FS_i &=& \big(K \:|\: i \in V\big) , & & FT_k&=& \big( V \:|\: k \in K\big). \nonumber
	\end{IEEEeqnarray}
	Note: for brevity we have omitted the parameters defined by the transitive closure
	operator, \ie $E^*$, $P_i^*$ and $F_i^*$.
	The big-$M$ value is calculated by
	\[
		M=t_{sum} + \beta_{n,1} + \sum_{i=1}^{n-1} \phi_{i,i+1} =22 + 3 + (0+3+3)= 31. \qed
	\]
\end{example}

\section{First Station-Based Formulation}
\label{sec:mip:fsbf}
The first mixed integer program
of the \sua{2} that we present is a station-based
formulation, and for brevity we denote
this formulation by \fsbf{2}.
The decisions of a station-based formulation are related to the stations
along the assembly line.

The definitions of the decision variables in this model
have been presented in Table \ref{tab:mip:fsbfVars}.
The primary decision is denoted by $c$, which is a continuous
variable indicating the cycle time of the assembly line.
The start times, $s_i$, of each task denote when a task
begins execution relative to the launch time of the station where
task $i$ is performed at.
The assignment decisions between the tasks and stations
are encoded by variables $x_{ik}$.
Binary variables $y_{ij}$ and $z_{ij}$ are true if task $j$
follows task $i$ in the forward and backward station load respectively.
% Binary variable $o_{ik}$ is straightforward so we refer the reader 
% to Table \ref{tab:mip:fsbfVars}.
The continuous variable $w_i$ denotes the station number of task $i$.

\begin{table}[tbp]
	\def\arraystretch{1.1}
	\centering
	\caption{Decision variables of FSBF-2}
	\vspace{2mm}
	\begin{tabular}{lp{0.8\textwidth}}
		\toprule
		Variable & Definition \\\midrule\midrule
		$c$ & cycle time of the assembly line, continuous\\
		$s_i$ & start time of task $i\in V$, continuous \\
		$x_{ik}$ & 1 iff task $i \in V$ is assigned to station $k \in FS_i$ \\
		$y_{ij}$ & 1 iff task $i \in V$ is followed directly by task $j \in F_i^\phi$ in the forward station load \\
		$z_{ij}$ & 1 iff task $i \in V$ is the last task completed and task $j \in F_i^\beta$ is the first in the same station \\
		$o_{ik}$ & 1 iff task $i \in V$ is the last task processed on station $k \in FS_i$ \\
		$w_i$ & encodes the number of the station task $i\in V$ is assigned, continuous \\
		\bottomrule
	\end{tabular}
	\label{tab:mip:fsbfVars}
\end{table}

\begin{IEEEeqnarray}{rClCl}
	\IEEEeqnarraymulticol{3}{l}{\text{\fsbf{2}: Min } c} & \hspace{4mm} & \label{eq:mip:fsbf1}\\[\eqnv]
	\text{s.t. } & \hspace{4mm} & \sum_{k\in FS_i} x_{ik} = 1 & & \forall i \in V \label{eq:mip:fsbf3}\\[\eqnv]
	& & \sum_{k\in FS_i} k\cdot x_{ik} = w_i & & \forall i \in V \label{eq:mip:fsbf4}\\[\eqnv]
	& & \sum_{j\in F_i^\phi} y_{ij} + \sum_{j\in F_i^\beta}z_{ij} = 1 & & \forall i \in V \label{eq:mip:fsbf5}\\[\eqnv]
	& & \sum_{i\in P_j^\phi} y_{ij} + \sum_{i\in P_j^\beta}z_{ij} = 1 & & \forall j \in V \label{eq:mip:fsbf6}\\[\eqnv]
	& & o_{ik} \leq x_{ik} & & \forall i \in V,~ k \in FS_i \label{eq:mip:fsbf7}\\[\eqnv]
	& & \sum_{j\in F_i^\beta} z_{ij} \leq \sum_{k\in FS_i} o_{ik} & & \forall i \in V \label{eq:mip:fsbf8}\\[\eqnv]
	& & \sum_{i\in FT_k} o_{ik} \leq 1 & & \forall k \in K \label{eq:mip:fsbf9}\\[\eqnv]
	& & w_j - w_i \leq M\cdot(1-y_{ij}) & & \nonumber\\[\smolEqnv]
	& & \text{ and }~w_i - w_j \leq M\cdot(1-y_{ij}) & & \forall i \in V,~ j \in F_i^\phi \label{eq:mip:fsbf10}\\[\eqnv]
	& & w_j - w_i \leq M\cdot(1-z_{ij}) & & \nonumber\\[\smolEqnv]
	& & \text{ and }~w_i - w_j \leq M\cdot(1-z_{ij}) & & \forall i \in V,~ j \in F_i^\beta \label{eq:mip:fsbf11}\\[\eqnv]
	& & \sum_{i\in FT_k} t_i\cdot x_{ik}\leq c & & \forall k \in K \label{eq:mip:fsbf12}\\[\eqnv]
	& & s_i+t_i+\sum_{j\in F_i^{\beta}} \beta_{ij}\cdot z_{ij}\leq c & & \forall i \in V \label{eq:mip:fsbf13}\\[\eqnv]
	& & \sum_{i\in V}\sum_{j\in F_i^\beta}z_{ij}\geq m & &  \label{eq:mip:fsbf14}\\[\eqnv]
	& & s_i+c\cdot(w_i-w_j)+t_i+\phi_{ij}\cdot y_{ij} \leq s_j & & \forall (i,j) \in E \label{eq:mip:fsbf15}\\[\eqnv]
	& & s_i+(t_i+\phi_{ij})+(c+\phi_{ij})\cdot (y_{ij}-1) \leq s_j & & \forall i \in V,~ j \in F_i^\phi\setminus F_i \label{eq:mip:fsbf16}\\[\eqnv]
	& & \ul{c}\leq c \leq \bar{c} & &  \label{eq:mip:fsbf17}\\[\eqnv]
	& & s_i \geq 0 & & \forall i \in V \label{eq:mip:fsbf18}\\[\eqnv]
	& & x_{ik}\in\{0,1\} & & \forall i \in V,~ k \in FS_i \label{eq:mip:fsbf19}\\[\eqnv]
	& & y_{ij}\in\{0,1\} & & \forall i \in V,~ j \in F_i^\phi \label{eq:mip:fsbf20}\\[\eqnv]
	& & z_{ij}\in\{0,1\} & & \forall i \in V,~ j \in F_i^\beta \label{eq:mip:fsbf21}\\[\eqnv]
	& & o_{ik} \geq 0 & & \forall i \in V,~ k \in FS_i  \label{eq:mip:fsbf22}\\[\eqnv]
	& & w_i \in \{1,2,\ldots,m\} & & \forall i \in V \label{eq:mip:fsbf23}
\end{IEEEeqnarray}

The objective function (\ref{eq:mip:fsbf1}) minimizes the cycle time
(as with all programs presented in this chapter), 
(\ref{eq:mip:fsbf3}--\ref{eq:mip:fsbf16}) define the main constraints
on this formulation and the last constraints (\ref{eq:mip:fsbf17}--\ref{eq:mip:fsbf23})
are non-functional defining the domains of our decision variables.
We now detail the purpose of each of the main constraints.

Constraint (\ref{eq:mip:fsbf3}) ensures that each task is assigned exactly one station,
(\ref{eq:mip:fsbf4}) encode the station numbers of the variables $w_i$,
constraints (\ref{eq:mip:fsbf5}) and (\ref{eq:mip:fsbf6}) enforce that
each task has exactly one follower and one predecessor respectively (in either
load direction),
(\ref{eq:mip:fsbf7}) requires that a task can only be the last task of a station
if it has been assigned to that station,
constraint (\ref{eq:mip:fsbf8}) ensures that the number of last tasks is at least
the number of backward setups,
(\ref{eq:mip:fsbf9}) enforces that each station has at maximum one last task.
The four inequalities of constraints (\ref{eq:mip:fsbf10}) and (\ref{eq:mip:fsbf11})
encode a disjunction using a big-$M$ value to ensure that only tasks in the 
the same station can be a part of the same sequence.
Knapsack constraint (\ref{eq:mip:fsbf12}) strengthens the formulation by
requiring all tasks be completed by the cycle time,
(\ref{eq:mip:fsbf13}) enforces the backward setup times to bound the cycle time below,
(\ref{eq:mip:fsbf14}) ensures that there are at least $m$ backward setup times,
constraint (\ref{eq:mip:fsbf15}) requires that the precedence relations are satisfied
within stations and between stations using the cycle time as a big-$M$ value
and finally (\ref{eq:mip:fsbf16}) enforces that the forward setup times are satisfied
in the forward station load of each station.

Both a lower and upper bound on the cycle time is required as input for
this model, so these are calculated as follows,
\begin{align}
	\ul{c} &=\max\left\{\: t_{max}',\:\left\lceil \frac{t_{sum}}{m} \right\rceil \:\right\} \label{eq:mip:cLB}\\
	\bar{c} &=t_{sum} + \beta_{n,1} + \sum_{i=1}^{n-1} \phi_{i,i+1}, \label{eq:mip:cUB}
\end{align}
where $t_{max}'=\max_{i\in V}(t_i+\beta_{ii})$.
The theoretical lower bound given by (\ref{eq:mip:cLB})
is the larger value between the maximum possible processing of any singular task
in a station and the equal fractional distribution of all processing times across 
all $m$ stations.

We note that the upper bound on the cycle time, given by (\ref{eq:mip:cUB}), need only be the cycle
time of some feasible solution.
The precedence graph is numbered topologically,
thus a precedence-feasible sequencing
of all tasks within a single station is 1,2,\ldots,n.
The upper bound on the cycle time is trivially
the sum of all processing times together with the setup times
resulting from the above sequencing.
As this assignment and sequence is valid, the upper bound is feasible.

The number of variables and constraints are both bounded by $\mathcal{O}(n^2)$.
We note here that this formulation of the \sua{2} requires a number
of disjunctive constraints, which can have an
undesirable impact on the ability of MIP solvers to
handle the problem.
This is due to MIPs needing to formulate disjunctive
constraints with a big-$M$ value which leads to weak
linear relaxations.	
Constraints (\ref{eq:mip:fsbf10}), (\ref{eq:mip:fsbf11}), 
(\ref{eq:mip:fsbf15}) and (\ref{eq:mip:fsbf16}) all encode disjunctions.

\section{Second Station-Based Formulation}
\label{sec:mip:ssbf}
The next formulation is abbreviated as \ssbf{}
and in this case the number of 
variables is bounded by $\mathcal{O}(n^3)$.
The decision variables in this model are similar
to \fsbf{2} except for a few differences which
are presented in Table \ref{sec:mip:ssbf}.

\begin{table}[tbp]
	\def\arraystretch{1.1}
	\centering
	\caption{Decision variables of SSBF-2}
	\vspace{2mm}
	\begin{tabular}{lp{0.8\textwidth}}
		\toprule
		Variable & Definition \\\midrule\midrule
		$g_{ijk}$ & 1 iff $i\in V$ is followed directly by task $j \in F_i^\phi$ in the forward station load on station $k\in FS_i$ \\
		$h_{ijk}$ & 1 iff $i\in V$ is the last task of station $k\in FS_i$ and $j\in F_i^\beta$ is the first \\
		$r_i$ & ordering variable which represents the priority of task $i\in V$ in a station's sequence of tasks\\
		\bottomrule
	\end{tabular}
	\label{tab:mip:ssbfVars}
\end{table}

\begin{IEEEeqnarray}{rClCl}
	\IEEEeqnarraymulticol{3}{l}{\text{\ssbf{2}: Min } c} & \hspace{4mm} & \label{eq:mip:ssbf1}\\[\eqnv]
	\text{s.t. } & \hspace{1mm} & \IEEEeqnarraymulticol{3}{l}{(\ref{eq:mip:fsbf5}),~(\ref{eq:mip:fsbf6}),~(\ref{eq:mip:fsbf17}),~(\ref{eq:mip:fsbf19}),~(\ref{eq:mip:fsbf23})} \nonumber\\[\eqnv]
	& & \sum_{j\in FT_k \cap F_i^\phi}g_{ijk} +\sum_{j\in FT_k \cap F_i^\beta} h_{ijk} = x_{ik} & & \forall i \in V,~ k\in FS_i \label{eq:mip:ssbf3}\\[\eqnv]
	& & \sum_{i\in FT_k \cap P_j^\phi}g_{ijk} +\sum_{i\in FT_k \cap P_j^\beta} h_{ijk} = x_{ik} & & \forall j \in V,~ k\in FS_j \label{eq:mip:ssbf4}\\[\eqnv]
	& & \sum_{i\in FT_k} \sum_{j\in FT_k \cap F_i^\beta} h_{ijk} = 1 & & \forall k\in K \label{eq:mip:ssbf5}\\[\eqnv]
	& & (n-|F_i^*|-|P_j^*|)\cdot\left( \sum_{k\in FS_i \cap FS_j} g_{ijk} -1 \right) & & \nonumber\\[\smolEqnv]
	& & ~~+r_i +1 \leq r_j & & \forall i\in V,~ j\in F_i^\phi \label{eq:mip:ssbf6}\\[\eqnv]
	& & r_i +1 \leq r_j & & \forall (i,j) \in E \label{eq:mip:ssbf7}\\[\eqnv]
	& & w_i \leq w_j & & \forall (i,j) \in E \label{eq:mip:ssbf8}\\[\eqnv]
	& & \sum_{i\in FT_k} t_i \cdot x_{ik} +\sum_{i\in FT_k} \sum_{j\in FT_k \cap F_i^\phi} \phi_{ij} g_{ijk} & &  \nonumber\\[\smolEqnv]
	& & ~~+\sum_{i\in FT_k} \sum_{j\in FT_k \cap F_i^\beta} \beta_{ij}h_{ijk} \leq c & & \forall k \in K \label{eq:mip:ssbf9}\\[\eqnv]
	& & \sum_{i\in FT_k \setminus \{j\}} x_{ik} \leq (n-m+1)\cdot(1-h_{ijk}) & & \forall k \in K,~ j\in FT_k \label{eq:mip:ssbf10}\\[\eqnv]
	& & g_{ijk} \in \{0,1\} & & \forall k \in K,~ i\in FT_k,~ j\in FT_k \cap F_i^\phi \nonumber\\[\smolEqnv]
	& & & & \label{eq:mip:ssbf11}\\[\eqnv]
	& & h_{ijk} \in \{0,1\} & & \forall k \in K,~ i\in FT_k,~ j\in FT_k \cap F_i^\beta \nonumber\\[\smolEqnv]
	& & & & \label{eq:mip:ssbf12}\\[\eqnv]
	& & |P_i^*|+1 \leq r_i \leq n-|F_i^*| & & \forall i \in V \label{eq:mip:ssbf13}
\end{IEEEeqnarray}

Constraints (\ref{eq:mip:ssbf3}--\ref{eq:mip:ssbf4}) ensure each task has exactly one
successor and one predecessor in the cyclic sequence of the assigned station and
each cycle is forced to contain exactly one backward setup by constraint (\ref{eq:mip:ssbf5}).
Constraint set (\ref{eq:mip:ssbf6}--\ref{eq:mip:ssbf7}) establish the
precedence relations between the tasks within each station.
We note here that (\ref{eq:mip:ssbf6}) is a disjunctive constraint
and becomes inactive when tasks $i$ and $j$ are assigned
to different stations, \ie $g_{ijk}=0$.
Due to this, constraint (\ref{eq:mip:ssbf8}) is added to
ensure the precedence relations are satisfied between stations.
A knapsack constraint (\ref{eq:mip:ssbf9}) is again included 
to strengthen the formulation.
If a task is its own successor, \ie $h_{jjk}=1$, then
constraint (\ref{eq:mip:ssbf10}) requires that no other tasks are
assigned the same station as task $j$.
Constraints (\ref{eq:mip:ssbf11}--\ref{eq:mip:ssbf13}) are the non-functional
constraints defining the new decision variables domains.

This model needs to use two disjunctions to achieve
a correct representation of the problem.
These disjunctive constraints are
(\ref{eq:mip:ssbf6}) and (\ref{eq:mip:ssbf10}).

\section{Scheduling-Based Formulation}
\label{sec:mip:scbf}
We now demonstrate that the \sua{} can be formulated
exclusively as a scheduling problem, namely as the 
\scbf{}.
Taking this approach means we must again choose
a new way of encoding the decisions of the problem.
This leads us to the new decision variable $q_{ij}$
which takes the value 1 if and only if task $i\in V$ is processed
before task $j \in F_i^\phi$ along the assembly line.
Together with $q_{ij}$ we again use decision variables
$s_i$, $y_{ij}$ and $z_{ij}$ as defined previously.
Variable $r_i$ is also reused however now its value
represents the position of task $i$ in the full schedule of
tasks, rather than within a single cyclic sequence of 
a station.

\begin{IEEEeqnarray}{rClCl}
	\IEEEeqnarraymulticol{3}{l}{\text{\scbf{2}: Min } c} & \hspace{4mm} & \label{eq:mip:scbf1}\\[\eqnv]
	\text{s.t. } & \hspace{1mm} & \IEEEeqnarraymulticol{3}{l}{(\ref{eq:mip:fsbf3}),~(\ref{eq:mip:fsbf4}),~(\ref{eq:mip:fsbf17}),~(\ref{eq:mip:fsbf18}),~(\ref{eq:mip:fsbf20}),~(\ref{eq:mip:fsbf21}),~(\ref{eq:mip:ssbf7}),~(\ref{eq:mip:ssbf13})} \nonumber\\[\smolEqnv]
	& & q_{ij} + q_{ji} = 1 & & \forall i \in V,~ j \in V\setminus (P_i^*\cup F_i^*) \text{ with } i<j \nonumber\\[\smolEqnv]
	& & & & \label{eq:mip:scbf2}\\[\eqnv]
	& & r_i + 1 + (n-|F_i^*|-|P_j^*|) & & \nonumber\\[\smolEqnv]
	& & ~~\times (q_{ij}-1) \leq r_j & & \forall i \in V,~ j \in V\setminus (P_i^*\cup F_i^*) \text{ with } i\neq j \nonumber\\[\smolEqnv]
	& & & & \label{eq:mip:scbf3}\\[\eqnv]
	& & r_j -1 + (n-|F_j^*|-|P_j^*|-1) & & \nonumber\\[\smolEqnv]
	& & ~~\times (y_{ij}-1) \leq r_i & & \forall i\in V,~ j\in F_i^\phi \label{eq:mip:scbf4}\\[\eqnv]
	& & y_{ij} \leq q_{ij} & & \forall i\in V,~ j\in F_i^\phi\setminus F_i^* \label{eq:mip:scbf5}\\[\eqnv]
	& & z_{ji} \leq q_{ij} & & \forall i\in V,~ j\in P_i^\beta\setminus F_i^*\text{ with }i\neq j \label{eq:mip:scbf6}\\[\eqnv]
	& & s_i + (t_i + \phi_{ij}) & & \nonumber\\[\smolEqnv]
	& & ~~+(c+\phi_{ij})\cdot(y_{ij}-1)\leq s_j & & \forall i \in V,~ j\in F_i^\phi \label{eq:mip:scbf7}\\[\eqnv]
	& & \sum_{i\in V}\sum_{j\in F_i^\beta}z_{ij} = m & &  \label{eq:mip:scbf8}\\[\eqnv]
	& & q_{ij} \in \{0,1\} & & \forall i\in V,~ j\in V\setminus (P_i^* \cup F_i^*) \text{ with } i \neq j \nonumber\\[\smolEqnv]
	& & & & \label{eq:mip:scbf9}
\end{IEEEeqnarray}

Constraint (\ref{eq:mip:scbf2}) establishes an ordering
between two tasks $i$ and $j$ which are not related
by the precedence graph.
The transitive relation between the $q_{ij}$ variables
is enforced by (\ref{eq:mip:scbf3}) together with (\ref{eq:mip:ssbf7}) and (\ref{eq:mip:scbf2}).
Explicitly, the combination of these constraints guarantee that for
three tasks $i$, $j$ and $v$, we have the following,
\[
	[q_{ij}=1]\wedge[q_{jv}=1] \Rightarrow [q_{iv}=1].
\]
Constraints (\ref{eq:mip:scbf4}--\ref{eq:mip:scbf5}) with (\ref{eq:mip:scbf3})
enforce the following implication,
\[
	[y_{ij}] \Rightarrow [r_j=r_i +1].
\]
Further to this, the conjunction of these constraints also encodes
the forward setup times.
Constraint (\ref{eq:mip:scbf6}) ensures that backward setups 
are accounted for in the schedule defined by the $q_{ij}$
variables.

This formulation requires disjunctive constraints 
(\ref{eq:mip:scbf3}), (\ref{eq:mip:scbf4}) and (\ref{eq:mip:scbf7}).

\section{Valid Inequalities}
\label{sec:mip:validIneqs}
In \authciteb{Esmaeilbeigi2016} some
valid inequalities are proposed to improve the formulations
of the type-2 version of \sua{}.
As we mentioned previously, none of these valid inequalities were tested in their
paper as their concern was with the type-1 version of the problem.

The following two inequalities can be added to both \fsbf{2}
and \scbf{2}
\begin{IEEEeqnarray}{lCl}
	y_{ij} + y_{ji} \leq 1 & \hspace{4mm} & \forall i \in V, j \in (F_i^\phi \cap P_i^\phi) \label{eq:mip:valIneq1}\\[\eqnv]
	w_{ii} + \sum_{j\in F_i^\beta}w_{ij} + \sum_{j\in P_i^\beta}w_{ji} \leq 1 & \hspace{4mm} & \forall i \in V \label{eq:mip:valIneq2}
\end{IEEEeqnarray}

Inequality (\ref{eq:mip:valIneq1}) states task $i$ cannot be before and 
after another task $j$ in the forward station load.
Equivalently, inequality (\ref{eq:mip:valIneq2}) does the same for
the backward station load.
\citeauthor{Esmaeilbeigi2016} note that (\ref{eq:mip:valIneq1})
dominates (\ref{eq:mip:valIneq2}) and thus only the
first inequality was tested in the relevant formulations.

Let us denote the total setup time that occurs throughout all stations
by $S$.
The idle time of a station is the difference
between the line's cycle time and the station's workload; let the
total idle time across all stations be denoted by $I$.
So the total amount of time used by the assembly line
can be calculated by summing all processing times with the total
setup time and the total idle time.
This results in the following relation $t_{sum}+S+I=m\cdot c$.
This can be formulated as a valid inequality to tighten the
linear relaxation of the MIPs as follows
\begin{IEEEeqnarray}{lCl}
	t_{sum}+\sum_{i\in V} \sum_{j\in F_i^\phi}\phi_{ij}\cdot y_{ij}+\sum_{i\in V} \sum_{j\in F_i^\beta}\beta_{ij}\cdot z_{ij} \leq m\cdot c. \label{eq:mip:valIneq3}
\end{IEEEeqnarray}
Here we interpret the left-hand side of inequality (\ref{eq:mip:valIneq3})
as a lower bound on the line capacity and the right-hand side
as the tightest possible upper-bound.
Once again, this inequality cannot be added to \ssbf{2} due to
(\ref{eq:mip:valIneq3}) requiring the decision variables $y_{ij}$
and $z_{ij}$.

\section{Summary}
\label{sec:mip:summary}
% The three mixed-integer programs presented in this chapter
% have been untested by the literature.
When testing the viability of a Benders decomposition
approach to a problem,
it is common practice to compare the computational results against
the best known mixed-integer programming formulation of the full problem.
Although, due to the \sua{2} being relatively under-studied,
no computational results of an exact solution
approach are available.
As such, we will test the MIPs presented here in order to receive benchmark results.
These results will allow us to judge the quality of our Benders decomposition method,
which will be detailed in the next chapter.

