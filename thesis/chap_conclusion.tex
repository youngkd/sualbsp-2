%!TEX root = thesis-kdyoung.tex

% \begin{savequote}%[45mm]
% ---Essentially, all models are wrong, but some are useful.
% \qauthor{George Box}
% \end{savequote}

\chapter{Conclusion}
\label{chap:conc}

In this thesis we devised a logic-based Benders decomposition
of the \sua{2}.
The natural division between the assignment and scheduling
variables led us to believe that this approach would be
well-suited to the problem.
Our formulation resulted in a classical \sab{2}
being solved by the master problem while
a sub-problem for each station was created.
We considered multiple ways of formulating and solving the sequencing
problems that arose in each station.

Logic-based Benders decomposition allowed us to 
formulate a sub-problem as either a MIP or a constraint program and 
using its inference dual to generate Bender cuts.
Employing CP solving technology resulted in marginal
improvements to the runtime and solution quality when compared
to MIP sub-problems.
However the CP solver \chuffed was only able to 
operate on one core while the MIP solver \gurobi was able to utilize
all four cores.
The inclusion of the global constraint \cumu in the CP model
and investigating the possible search strategies of the CP solver,
both contributed to the CP model outperforming the
MIP sub-problem formulation.

Both feasibility and optimality Benders cuts were devised.
Solving each sub-problem to optimality allowed a much wider
range of cuts to be generated at each iteration of the Benders
algorithm.
We provided a detailed analysis into the advantages and disadvantages
of each possible cutting procedure and after performing experiments,
we were able to conclude that combining the global upper bound, $\mathcal{C}_{gb}$,
with our strongest inference cut, $\mathcal{C}_{iii}$, resulted
in the most effective cutting planes added to the relaxed master problem.

In summary, our decomposition approach was able to solve
100\% of the smallest class of instances in an average runtime of
0.41 seconds.
For the second class, which contained small-to-medium sized instances,
our method was able to solve 61.6\% of the instances, while also
finding superior gap values compared to pure MIP models solved with \gurobi.
In the case of largest class we considered, only 15.3\% of optimal solutions
were proven in the 30 minute time limit.
Of the 396 instances we tested which had up to 58 tasks and 31 stations,
the Benders decomposition was able to solve 51.5\% to optimality.

\section{Future Work}
\label{sec:conc:future}
The work and results we have reported on in this thesis
stand as only the very beginning of the research which
could be done into decomposition approaches to the \sua{}.
The logic-based Benders decomposition we devised can be viewed
simply as a framework for a solution procedure to this problem.
As such this framework presents a range of possible future research directions;
some of which we will now detail.

Firstly, applying our Benders decomposition to the type-1 problem
of \sua{} is a straight forward extension.
This will result in $m$ sub-problems needing to be solved
where $m$ is a variable, while each station has an upper bound
on its load.
We can again view these sequencing problems as asymmetric TSPs, however
now not only is the number of cities within each TSP unknown,
but the number of TSPs to be solved is also a decision.
As each sub-problem has a pre-defined upper bound, feasibility
cuts may be better suited in the type-1 case.

The final results we presented in the previous chapter tell us that
the vast majority of solve time in our decomposition is due to
the relaxed master problem.
We only considered a full mixed integer programming approach for the 
master however this was not exclusively necessary. 
Any procedure which can return an assignment of tasks to the stations and provides 
a lower bound on the cycle time would be applicable.
Thus, developing heuristic procedures to replace our current formulation of
the \rmp{} may be a fruitful direction of investigation.
In fact, it is very possible that implementing such a procedure --- even if it is 
relatively simple --- could significantly improve the results 
found here.

In this thesis, we considered few ways of formulating the MIP
version of the sub-problems.
Further investigation into valid inequalities of this program
or simply a different set of decision variables could lead
to better results.
Regarding the CP sub-problems, the lack of parallelization
of the \chuffed solver hindered its capabilities somewhat.
So when this feature becomes available, re-running the
experiments to check the reduction in runtime 
would provide a useful comparison between different
solution technologies.

Finally, we emphasise the limited amount of research which
has been done into the \sua{}.
This variant of the \albp{} presents a non-trivial layer of complexity
with a set of scheduling decisions,
however this consideration is applicable to a wide variety of assembly lines.
Thus, including setup costs in other well-known variants of the 
\albp{} would lead to a range of possible research directions.
